\documentclass[10pt,a4paper]{article}

\usepackage[backend=bibtex]{biblatex}
\addbibresource{biblio.bib}

\usepackage[utf8]{inputenc}
\usepackage[french]{babel}
\usepackage[T1]{fontenc}
\usepackage{amsmath}
\usepackage{amsfonts}
\usepackage{amssymb}
\begin{document}

\paragraph{}

On considère la forme de fonction séculaire suivante :

$$f(x) = \rho + \sum_{i=1}^n \frac{\zeta_i^2}{\delta_i - x} $$

Ses racines correspondent aux valeurs propre d'une matrice diagonale $D = diag(\delta_1,...\delta_n) $ à laquelle une perturbation de rang 1 a été appliquée : 
$$ D + \frac{zz^T}{\rho}$$

On pose $z = (\zeta_1,...\zeta_n)$

Il faut que $(\delta_1,...\delta_n)$ soit rangé par ordre croissant, donc $\delta_1$ est la plus petite VP et $\delta_n$ la plus grande.
La fonction $f$ a $n$ racines, $n - 1$ dans les intervalles $[\delta_k, \delta_{k+1}]$ et une dans $[\delta_n, + \infty]$ (si $\rho >0$ ce que l'on supposera pour toute la suite). On cherche à trouver ces racines de manière numérique en utilisant l'algorithme de Gragg, on résume dans la suite les formules pour les différentes itérations tirées de \cite{1}.

\paragraph{Tiré de la section 3.1 pages 12 et 13:}

\paragraph{Pour les racines intérieures}
En se plaçant à un $y$ donné, cherche à résoudre une équation du second degré ayant pour inconnu $\eta$ afin que $y + \eta$ soit plus proche de la racine que l'on cherche que $y$. 
En posant : 
$$ \Delta_k = \delta_k - y$$ 
$$ \Delta_ {k+1} = \delta_{k+1} - y$$ 
$$ a = (\Delta_k + \Delta_{k+1}) f(y) - \Delta_k \Delta_{k+1}f'(y)$$
$$ b = \Delta_k \Delta_{k+1} f(y)$$

$\eta$ est donné par : 

\begin{eqnarray}\label{discri1}
\eta = \frac{a - \sqrt{a^2 - 4bc}}{2c}~~ si~~ a \leq 0 \\
\eta = \frac{2b}{a + \sqrt{a^2 - 4bc}}~~ si~~ a > 0
\end{eqnarray}

\paragraph{Tiré de la section 3.3 page 15}:

\paragraph{}
$c$ est un degré de liberté qui nous est donné pour l'interpolation. Dans le cas de l'algorithme de Gragg, $c$ est choisi tel que l'interpolation rationnelle de la fonction séculaire en $y$ coïncide aussi avec la dérivée seconde de cette dernière en ce même point. La formule pour $c$ est alors : 

\begin{equation}\label{c_gragg}
c = f(y) - (\Delta_k + \Delta_{k+1}) f'(y) + \Delta_k \Delta_{k+1} \frac{f''(y)}{2}
\end{equation}

\paragraph{}
En mettant ensemble (\ref{discri1}) et (\ref{c_gragg}), on peut donc calculer $\eta$, reste à incrémenter $y$ en lui ajoutant $\eta$. C'est la formule d'itération de Gragg pour les racines intérieures.


\paragraph{Pour la racine extérieure ($k=n$)}

Il faut adapter un peu la précédente procédure. On garde la même interpolation que pour $k=n-1$ donc ;

$$ \Delta_{n-1} = \delta_{n-1} - y$$ 
$$ \Delta_n = \delta_n - y$$ 
$$ a = (\Delta_{n-1} + \Delta_n) f(y) - \Delta_{n-1} \Delta_nf'(y)$$
$$ b = \Delta_{n-1} \Delta_n f(y)$$
$$c = f(y) - (\Delta_{n-1} + \Delta_n) f'(y) + \Delta_{n-1} \Delta_n \frac{f''(y)}{2}$$

Sauf que $\eta$ est donnée par une version modifiée de (\ref{discri1}) : 

\begin{eqnarray}\label{discri2}
\eta = \frac{a + \sqrt{a^2 - 4bc}}{2c}~~ si~~ a \geq 0 \\
\eta = \frac{2b}{a - \sqrt{a^2 - 4bc}}~~ si~~ a < 0
\end{eqnarray}

\paragraph{}
On peut là aussi calculer $\eta$ et ainsi l'ajouter à $y$ pour se rapprocher de la racine.


\paragraph{Choix des valeurs initiales : section 4 - Initial guesses pages 18 - 19 - 20}

Pour les racines intérieures : voir fin de la page 19 et formules (42), (43) et (44).

Pour la racine extérieure voir la page 21 (notamment partie avec la distinction de cas et les formules (46) et (47)).




\printbibliography

\end{document}